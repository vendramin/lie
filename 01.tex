\section{Lie algebras}

%\index{Algebra}
%Let $K$ be a field. In this course, an algebra over $K$ 
%is a pair $(A,m)$, where $A$ is a $K$-vector space and
%$m\colon A \times A \to A$, $(x,y) \mapsto xy$, is a bilinear map. 
%Note that we do not assume that the operation $m$ is associative, 
%nor that $a$ unit exists.

\begin{definition}
  \index{Lie algebra}
  \label{def:Lie}
  A \emph{Lie algebra} is a pair $(L,[-,-])$, where $L$ is a 
  $K$-vector space and $[-,-]\colon L\times L\to L$ is a 
  bilinear map such that 
  \begin{enumerate}
    \item $[x,x]=0$ for all $x\in L$, and 
    \item $[x,[y,z]]+[y,[z,x]]+[z,[x,y]]=0$ for all $x,y,z\in L$. 
  \end{enumerate}
\end{definition}

\index{Jacobi identity}
The first condition of Definition~\ref{def:Lie} 
is \emph{anticommutativity}. The second condition is 
the \emph{Jacobi identity}. 

\begin{exercise}
  Let $L$ be a Lie algebra. Prove that 
  $[x,y]=-[y,x]$ for all $x,y\in L$. Does 
  $[x,y]=-[y,x]$ for all
  $x,y\in L$ imply that $[x,x]=0$ for all $x\in L$? 
\end{exercise}

\begin{exercise}
  A Lie algebra $L$ is said to be \emph{abelian} if $[x,y]=0$ for all $x,y\in L$. Prove
  that a one-dimensional Lie algebra is abelian. 
\end{exercise} 

\begin{exercise}
  Let $L$ be a Lie algebra. Prove that 
  \[
    [x,[y,z]]=[[x,y],z]+[y,[x,z]]
  \]
  for all $x,y,z\in L$. 
\end{exercise}

\begin{example}
  Let $A$ be an associative algebra. Then $A$ together with the bracket
  $[a,b]=ab-ba$ is a Lie algebra. This Lie algebra is denoted by $L(A)$. 
\end{example}

\begin{exercise}
  Prove that $\R^{3\times 1}$ together 
  with the usual cross product 
  \[
    \begin{pmatrix}
      v_1\\
      v_2\\
      v_3
    \end{pmatrix}
    \times
    \begin{pmatrix}
      w_1\\
      w_2\\
      w_3
    \end{pmatrix}
%    =
%    v\times w=\det\begin{pmatrix}
%      i & j & k\\
%      v_1 & v_2 & v_3\\
%      w_1 & w_2 & w_3
%    \end{pmatrix}
    =\begin{pmatrix}
    v_2w_3-v_3w_2\\
    v_3w_1-v_1w_3\\
    v_1w_2-v_2w_1
  \end{pmatrix}
  \]
  is a (real) Lie algebra. 
\end{exercise}

\begin{exercise}
    If $L$ and $L_1$ are Lie algebras, then 
    $L\oplus L_1$ is a Lie algebra with
    $[(x,x_1),(y,y_1)]=([x,y),(x_1,y_1)]$ for $x,y\in L$ and
    $x_1,y_1\in L_1$. 
\end{exercise}

We will mainly work with finite-dimensional complex Lie algebras.

\begin{example}[general linear Lie algebra]
\index{General linear Lie algebra}
    Let $V$ be a finite-dimensional vector space and 
    $\gl(V)$ be the set of linear maps $V\to V$. Then 
    $\gl(V)$ with $[x,y]=xy-yx$ is a Lie algebra. 
\end{example}

A matrix version of the previous example: We write $\gl(n,\C)$ 
to denote the vector space of all $n\times n$ complex 
matrices with Lie bracket $[x,y]=xy-yx$. The vector space
$\gl(n,\C)$ has a basis $\{e_{ij}:1\leq i,j\leq n\}$, where
\[
(e_{ij})_{kl}=\begin{cases} 
    1 & \text{if $(i,j)=(k,l)$},\\
    0 & \text{otherwise}.
    \end{cases}
\]

\begin{exercise}
    Compute $[e_{ij},e_{ik}]$.
\end{exercise}

\begin{example}[special linear Lie algebra]
\index{Special linear Lie algebra}
    Let $\sl(n,\C)$ be the subspace of $\gl(n,\C)$ consisting
    of all matrices with trace zero. 
\end{example}

\begin{exercise}
    Find a basis of $\sl(n,\C)$. 
\end{exercise}

%\begin{example}
%    Let $\bl(n,\C)$ be the subspace of all upper triangular matrices
%    in $\gl(n,\C)$. Then $\bl(n,\C)$ is a Lie algebra. 
%\end{example}

\begin{definition}
  \index{Subalgebra}
  Let $L$ be a Lie algebra. A subspace $A$ 
  of $L$ is a Lie \emph{subalgebra} of $L$ if $[A,A]\subseteq A$. 
\end{definition}

Of course, $\sl(n,\C)$ is a subalgebra of $\gl(n,\C)$. 

\begin{definition}
  \index{Ideal}
  Let $L$ be a Lie algebra. A subspace $I$ of $L$ 
  is an \emph{ideal} of $L$ if 
  $[I,L]\subseteq I$. 
\end{definition}

Note that if $I$ is an ideal of $L$, then 
$[L,I]\subseteq I$. Note also that 
every ideal is a subalgebra. 

Trivial examples of ideals of a Lie algebra $L$ are
$\{0\}$ and $L$.

\begin{example}
\index{Center!of a Lie algebra}
    Let $L$ be a Lie algebra. Then 
    the \emph{center} 
    \[
    Z(L)=\{x\in L:[x,y]=0\text{ for all $y\in L$}\}.
    \]
    is an ideal of $L$. 
\end{example}

\begin{exercise}
    Compute $Z(\sl(n,\C))$. 
\end{exercise}

\begin{exercise}
    Prove that $\sl(2,\C)$ has no non-trivial ideals. 
\end{exercise}

One easily checks that $\sl(n,\C)$ is an ideal of $\gl(n,\C)$. In fact, 
an ideal is always a subalgebra. The converse is not true. 
Can you find an example?



\begin{example}
\index{Derived algebra!of a Lie algebra}
    Let $L$ be a Lie algebra. 
    The \emph{derived algebra} $[L,L]$
    consists of all linear combinations of commutators $[x,y]$ 
    is an ideal of $L$. 
\end{example}




\begin{exercise}
  Let $L$ be a Lie algebra and $I$ be an ideal of $L$. 
  Prove that $L/I$ is a Lie algebra with
  $[x+I,y+I]=[x,y]+I$. 
\end{exercise}

\begin{definition}
  Let $L$ and $L_1$ be Lie algebras. A linear map 
  $f\colon L\to L_1$ is a \text{homomorphism} of Lie algebras
  if $f([x,y])=[f(x),f(y)]$ for all $x,y\in L$. 
\end{definition}

As usual, an isomorphism between Lie algebras will be
a bijective homomorphism of Lie algebras. 

\begin{example}
    Let $L$ and $L_1$ be Lie algebras. The canonical injections
    $L\to L\oplus L_1$ and $L_1\to L_\oplus L_1$ and
    the canonical surjections $L\oplus L_1\to L$ and 
    $L\oplus L_1\to L_1$ are Lie algebras homomorphisms.  
\end{example}

\begin{example} 
  Let $L$ be a Lie algebra and $I$ an ideal of $L$. Then 
  the \emph{canonical map} $\pi\colon L\to L/I$, $x\mapsto x+I$, 
  is a homomorphism. 
\end{example}

\begin{example}
    Let $L$ be a Lie algebra. The \emph{opposite Lie algebra} 
    $L^{\op}$ is the vector space $L$ with 
    $[x,y]^{\op}=-[x,y]$. Then $L\to L^{\op}$, $x\mapsto -x$, 
    is an isomorphism of Lie algebras.
\end{example}

\begin{exercise}
    Let $f\colon L\to L_1$ be a Lie algebra homomorphism. Prove
    that the \emph{kernel} of $f$, 
    $\ker f=\{x\in L:f(x)=0\}$ is an ideal
    of $L$, and that the \emph{image} of $f$ 
    is a subalgebra of $L_1$. 
\end{exercise}

\begin{example}
\index{Adjoint homomorphism}
    Let $L$ be a Lie algebra. 
    The \emph{adjoint homomorphism} is the map 
    \[
    \ad\colon L\to\gl(L),\quad
    (\ad x)(y)=[x,y].
    \]
\end{example}



\begin{exercise} 
Let $V$ be a finite-dimensional vector space 
and $\mathcal{L}(V,V)$ the set of linear operators on $V$. Then 
$\mathcal{L}(V,V)$ is an associative algebra...
Prove the following statements: 
\begin{enumerate}
  \item $\C\id_V$ is an ideal of $\gl(V)$. 
  \item $\gl(V)/\C\id_V\simeq\sl(V)$. 
\end{enumerate}
\end{exercise} 

\index{Derivation}
Let $A$ be a vector space admitting a bilinear operation $A\times A\to A$,
$(a,b)\mapsto ab$. (We do not assume here that the binary operation is
associative.) A \emph{derivation} is a linear map $\partial\colon A\to A$ such
that $\partial(ab)=(\partial a)b+a(\partial b)$ for all $a,b\in A$. 

\begin{exercise}
  Let $L$ be a Lie algebra and $x\in L$. Prove that 
  $y\mapsto [x,y]$ is a derivation of $L$. 
\end{exercise}

\begin{example}
  The set $\operatorname{Der}(A)$ of derivation of $A$ 
  is a Lie subalgebra of $\gl(A)$. 
\end{example}

\begin{exercise}
    Let $f\colon L\to L_1$ be a Lie algebra homomorphism.
    Prove that $f/\ker f\simeq f(L)$. 
\end{exercise}

\begin{definition}
\index{Simple Lie algebra}
    A Lie algebra $L$ is said to be \emph{simple} if 
    $[L,L]\ne\{0\}$ and $\{0\}$ and $L$ are the only ideals of $L$. 
\end{definition}

If $L$ is a simple Lie algebra, then $Z(L)=\{0\}$ and $L=[L,L]$. 

\begin{exercise}
    Prove that every simple Lie algebra is isomorphic to 
    a linear Lie algebra. 
\end{exercise}



\begin{bonus}
\index{Witt Lie algebra}
  
\end{bonus}

\begin{bonus}
\index{Virasoro Lie algebra}
\end{bonus}



