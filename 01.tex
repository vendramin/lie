\section{Lie algebras}

\index{Algebra}
Let $K$ be a field. In this course, an algebra over $K$ 
is a pair $(A,m)$, where $A$ is a $K$-vector space and
$m\colon A \times A \to A$, $(x,y) \mapsto xy$, is a bilinear map.  We do not
assume that $m$ is associative, nor that a unit exists. By imposing additional
conditions on the binary operation, we obtain different kinds of algebras, such
as associative algebras and Lie algebras.

\begin{definition}
  \index{Lie algebra}
  \label{def:Lie}
  A \emph{Lie algebra} is a pair $(L,[-,-])$, where $L$ is a 
  $K$-vector space and $[-,-]\colon L\times L\to L$ is a 
  bilinear map such that 
  \begin{enumerate}
    \item $[x,x]=0$ for all $x\in L$, and 
    \item $[x,[y,z]]+[y,[z,x]]+[z,[x,y]]=0$ for all $x,y,z\in L$. 
  \end{enumerate}
\end{definition}

\index{Jacobi identity}
The first condition of Definition~\ref{def:Lie} 
is \emph{anticommutativity}. The second condition is 
the \emph{Jacobi identity}. 

\begin{exercise}
  Let $L$ be a Lie algebra. Prove that 
  $[x,y]=-[y,x]$ for all $x,y\in L$. Does 
  $[x,y]=-[y,x]$ for all
  $x,y\in L$ imply that $[x,x]=0$ for all $x\in L$? 
\end{exercise}

\begin{exercise}
  \index{Lie algebra!abelian}
  A Lie algebra $L$ is said to be \emph{abelian} if $[x,y]=0$ for all $x,y\in L$. Prove
  that one-dimensional Lie algebras are abelian. 
\end{exercise} 

\begin{exercise}
  \label{xca:Jacobi}
  Let $L$ be a Lie algebra. Prove that 
  \[
    [x,[y,z]]=[[x,y],z]+[y,[x,z]]
  \]
  for all $x,y,z\in L$. 
\end{exercise}

\begin{example}
  Let $A$ be an associative algebra. Then $A$ together with the bracket
  $[a,b]=ab-ba$ is a Lie algebra. Checking the Jacobi identity is
  straightforward:
  \begin{align*}
    [x,[y,z]]+[y,[z,x]]+[z,[x,y]]
    =x(yz-zy)&-(yz-zy)x\\
             &+y(zx-xz)-(zx-xz)y\\
             &+z(xy-yx)-(xy-yx)z=0.
  \end{align*} 
  This Lie algebra is denoted by $L(A)$. 
\end{example}

\begin{exercise}
  Prove that $\R^{3\times 1}$ together 
  with the usual cross product 
  \[
    \begin{pmatrix}
      v_1\\
      v_2\\
      v_3
    \end{pmatrix}
    \wedge 
    \begin{pmatrix}
      w_1\\
      w_2\\
      w_3
    \end{pmatrix}
%    =
%    v\times w=\det\begin{pmatrix}
%      i & j & k\\
%      v_1 & v_2 & v_3\\
%      w_1 & w_2 & w_3
%    \end{pmatrix}
    =\begin{pmatrix}
    v_2w_3-v_3w_2\\
    v_3w_1-v_1w_3\\
    v_1w_2-v_2w_1
  \end{pmatrix}
  \]
  is a (real) Lie algebra. 
\end{exercise}

\begin{exercise}
    If $L$ and $L_1$ are Lie algebras, then 
    $L\oplus L_1$ is a Lie algebra with
    $[(x,x_1),(y,y_1)]=([x,y],[x_1,y_1])$ for $x,y\in L$ and
    $x_1,y_1\in L_1$. 
\end{exercise}

We will mainly work with finite-dimensional complex Lie algebras.

\begin{example}
\index{General linear Lie algebra}
    Let $V$ be a finite-dimensional vector space and 
    $\gl(V)$ be the set of linear maps $V\to V$. Then 
    $\gl(V)$ with $[x,y]=xy-yx$ is a Lie algebra. This Lie algebra is called 
    the \emph{general Lie algebra}.
\end{example}

\index{$\gl(n,\C)$}
A matrix version of the previous example: We write $\gl(n,\C)$ 
to denote the vector space of all $n\times n$ complex 
matrices with Lie bracket $[x,y]=xy-yx$. The vector space
$\gl(n,\C)$ has a basis $\{e_{ij}:1\leq i,j\leq n\}$, where
\[
(e_{ij})_{kl}=\begin{cases} 
    1 & \text{if $(i,j)=(k,l)$},\\
    0 & \text{otherwise}.
    \end{cases}
\]

\begin{exercise}
  \label{xca:formulas}
  Prove that $[e_{ij},e_{ik}]=\delta_{j,k}e_{il}-\delta_{i,l}e_{kj}$.
\end{exercise}

\begin{example}
\index{$\sl(n,\C)$}
\index{Special linear Lie algebra}
    Let $\sl(n,\C)$ be the subspace of $\gl(n,\C)$ consisting
    of all matrices with trace zero. This Lie algebra is called the
    \emph{special linear algebra}. 
\end{example}

\begin{exercise}
    Find a basis of $\sl(n,\C)$. 
\end{exercise}

For example, a basis of $\sl(2,\C)$ is given by 
\[
  h=\begin{pmatrix}
  1&0\\
  0&-1
\end{pmatrix},
\quad 
e=\begin{pmatrix}
  0&1\\
  0&0
\end{pmatrix},
\quad 
f=\begin{pmatrix}
  0&0\\
  1&0
\end{pmatrix}.
\]
Note that
\[
[h,e]=2e,\quad
[h,f]=-2f,\quad
[e,f]=h.
\]

%\begin{example}
%    Let $\bl(n,\C)$ be the subspace of all upper triangular matrices
%    in $\gl(n,\C)$. Then $\bl(n,\C)$ is a Lie algebra. 
%\end{example}

\begin{definition}
  \index{Subalgebra}
  Let $L$ be a Lie algebra. A subspace $A$ of $L$ is a Lie \emph{subalgebra} of
  $L$ if $[A,A]\subseteq A$, that is $[a,a]\in A$ for all $a\in A$. 
\end{definition}

Of course, $\sl(n,\C)$ is a subalgebra of $\gl(n,\C)$. 

\begin{definition}
  \index{Ideal}
  Let $L$ be a Lie algebra. A subspace $I$ of $L$ 
  is an \emph{ideal} of $L$ if 
  $[I,L]\subseteq I$, that is $[a,x]\in I$ for all $a\in I$ and $x\in L$. 
\end{definition}

Note that if $I$ is an ideal of $L$, then 
$[L,I]\subseteq I$. Note also that 
every ideal is a subalgebra. 

Trivial examples of ideals of a Lie algebra $L$ are
$\{0\}$ and $L$.

\begin{example}
\index{Center!of a Lie algebra}
    Let $L$ be a Lie algebra. Then 
    the \emph{center} 
    \[
    Z(L)=\{x\in L:[x,y]=0\text{ for all $y\in L$}\}.
    \]
    is an ideal of $L$. 
\end{example}

\begin{exercise}
\label{xca:center_sln}
    Compute $Z(\sl(n,\C))$. 
\end{exercise}

For example, to compute the center of $\sl(2,\C)$, we proceed as follows: 
Let 
\[
  x=\begin{pmatrix}a&b\\c&-a\end{pmatrix}=ah+be+cf
\]
be a central element. Then
\[
  0=[x,e]=a[h,e]+b[e,e]+c[f,e]=2ae-ch.
\]
Since $e$ and $h$ are linearly independent, $a=c=0$. Now 
$[x,h]=0$ implies that $b=0$. Therefore
$x=\begin{pmatrix}0&0\\0&0\end{pmatrix}$.

One easily checks that $\sl(n,\C)$ is an ideal of $\gl(n,\C)$. In fact, an
ideal is always a subalgebra. The converse is not true.  Can you find an
example?

\begin{example}
\index{Derived algebra!of a Lie algebra}
    Let $L$ be a Lie algebra. 
    The \emph{derived algebra} $[L,L]$
    consists of all linear combinations of commutators $[x,y]$ 
    is an ideal of $L$. 
\end{example}

\begin{exercise}
  Let $L$ be a Lie algebra and $I$ be an ideal of $L$. 
  Prove that $L/I$ is a Lie algebra with
  $[x+I,y+I]=[x,y]+I$. 
\end{exercise}

\begin{definition}
  \index{Homomorphism}
  Let $L$ and $L_1$ be Lie algebras. A linear map 
  $f\colon L\to L_1$ is a \text{homomorphism} of Lie algebras
  if $f([x,y])=[f(x),f(y)]$ for all $x,y\in L$. 
\end{definition}

As usual, an isomorphism between Lie algebras will be
a bijective homomorphism of Lie algebras. 

\begin{example}
    Let $L$ and $L_1$ be Lie algebras. The canonical injections
    $L\to L\oplus L_1$ and $L_1\to L\oplus L_1$ and
    the canonical surjections $L\oplus L_1\to L$ and 
    $L\oplus L_1\to L_1$ are Lie algebras homomorphisms.  
\end{example}

\begin{example} 
  Let $L$ be a Lie algebra and $I$ an ideal of $L$. Then 
  the \emph{canonical map} $\pi\colon L\to L/I$, $x\mapsto x+I$, 
  is a homomorphism. 
\end{example}

\begin{example}
    Let $L$ be a Lie algebra. The \emph{opposite Lie algebra} 
    $L^{\op}$ is the vector space $L$ with 
    $[x,y]^{\op}=-[x,y]$. Then $L\to L^{\op}$, $x\mapsto -x$, 
    is an isomorphism of Lie algebras.
\end{example}

\begin{exercise}
    Let $f\colon L\to L_1$ be a Lie algebra homomorphism. Prove 
    the following statements: 
    \begin{enumerate} 
      \item The \emph{kernel} $\ker f=\{x\in L:f(x)=0\}$ of $f$ is an ideal
        of $L$. 
      \item The \emph{image} $f(L)$ of $f$ is a subalgebra of $L_1$. 
      \item $L/\ker f\simeq f(L)$. 
    \end{enumerate}
\end{exercise}

\begin{example}
\index{Adjoint homomorphism}
    Let $L$ be a Lie algebra. 
    The \emph{adjoint homomorphism} is the map 
    \[
    \ad\colon L\to\gl(L),\quad
    (\ad x)(y)=[x,y].
    \]
\end{example}

\begin{exercise}
  \label{xca:simple=>linear}
  Let $L$ be a simple Lie algebra. 
  Prove that $L$ is isomorphic to a subalgebra of $\gl(L)$. 
\end{exercise}

% Take the adjoint homomorphisms $L\to\gl(L)$. 

\begin{exercise} 
  \label{xca:gl(V)}
Let $V$ be a complex finite-dimensional vector space.
Prove the following statements: 
\begin{enumerate}
  \item $\C\id_V$ is an ideal of $\gl(V)$. 
  \item $\gl(V)/\C\id_V\simeq\sl(V)$. 
\end{enumerate}
\end{exercise} 

\index{Derivation}
A \emph{derivation} of an algebra $A$ is a linear map $\partial\colon A\to A$
such that $\partial(ab)=(\partial a)b+a(\partial b)$ for all $a,b\in A$. 

\begin{example}
Let $L$ be a Lie algebra and $x\in L$. Exercise~\ref{xca:Jacobi} 
implies that the map 
$y\mapsto [x,y]$ is a derivation of $L$. 
\end{example}

\begin{exercise}
Let $A$ be an algebra. Prove that the 
set $\operatorname{Der}(A)$ of derivations of $A$ 
is a Lie subalgebra of $\gl(A)$. 
\end{exercise}

%\begin{exercise}
%    Let $f\colon L\to L_1$ be a Lie algebra homomorphism.
%    Prove that $f/\ker f\simeq f(L)$. 
%\end{exercise}

\begin{definition}
\index{Simple Lie algebra}
    A Lie algebra $L$ is said to be \emph{simple} if 
    $[L,L]\ne\{0\}$ and $\{0\}$ and $L$ are the only ideals of $L$. 
\end{definition}

If $L$ is a simple Lie algebra, then $Z(L)=\{0\}$ and $L=[L,L]$. 

\begin{example}
  Let us show that $\sl(2,\C)$ is simple. Let $I$ be a non-zero ideal of
  $\sl(2,\C)$ and $x=ah+be+cf\in I$ be a non-zero element. Then
  \begin{align*}
    [h,x] = a[h,h]+b[h,e]+c[h,f]=2be-2cf\in I
  \end{align*}
  and hence $[h,[h,x]]=4be+4cf\in I$.

  If $a\ne 0$, then $4x-[h,[h,x]]=4ah\in I$ and hence 
  $h\in I$. Now that $h\in I$, it follows that $I=\sl(2,\C)$, as 
  $e=(1/2)[h,e]\in I$ and $f=(1/2)[h,f]\in I$. 

  If $a=0$, then $[e,x]=c[e,f]=ch\in I$. If $c\ne 0$, then $h\in I$ and it
  follows that $I=\sl(2,\C)$. If $c=0$, then $x=be\in I$. Thus $[e,f]=h\in I$
  and therefore $I=\sl(2,\C)$. 
\end{example}

%\begin{exercise}
%    Prove that every simple Lie algebra is isomorphic to 
%    a linear Lie algebra. 
%\end{exercise}

\begin{bonus}
\index{Witt Lie algebra}
Let $A=\C[X,X^{-1}]$. Prove the following statements:
\begin{enumerate}
  \item For $i\in\Z$, $d_i=X^i\frac{\partial}{\partial X}$ is a derivation.
  \item $\{d_i:i\in\Z\}$ is a basis of $\operatorname{Der}(A)$. 
  \item $[d_i,d_j]=(j-i)d_{i+j}$. 
  \item $\operatorname{Der}(A)$ is a Lie algebra. 
\end{enumerate}
The Lie algebra $\operatorname{Der}(A)$ is called the \emph{Witt Lie algebra}.   
\end{bonus}

\begin{bonus}
\index{Virasoro Lie algebra}
Let $V$ be the complex vector space with basis $\{e_i:i\in\Z\}\cup\{c\}$.
Define a bracket on the basis of $V$ by 
\[
  [e_i,e_j]=(j-i)e_{i+j}+\delta_{i,-j}\frac{i^3-i}{12}c,
\]
where
\[
  \delta_{x,y}=\begin{cases}
  1&\text{if $x=y$},\\
  0&\text{otherwise},
\end{cases}
\]
is the Kronecker function, and $[c,V]=0$. Prove the following statements:
\begin{enumerate}
  \item $V$ is a Lie algebra. 
  \item $\C c$ is an ideal of $V$. 
  \item $V/\C c$ is isomorphic to the Witt Lie algebra. 
\end{enumerate}
The Lie algebra $V$ is called the \emph{Virasoro Lie algebra}.   
\end{bonus}

\section{Representations}

\begin{definition}
  Let $L$ be a Lie algebra. A \emph{representation} of $L$ is a Lie algebra
  (over $K$) homomorphism $\rho\colon L\to\gl(V)$, where $V$ is a
  finite-dimensional vector space (over $K$). 
\end{definition}

\begin{example}
  \index{Trivial representation}
  Let $L$ be a Lie algebra and $V$ a vector space. 
  The \emph{trivial representation}
  is the representation $L\to\gl(V)$, $x\mapsto 0$. 
\end{example}

\begin{example}
  \index{Tautological representation}
  Let $V$ be a vector space. Then the identity map 
  $\id\colon\gl(V)\to\gl(V)$ 
  is a representation. 
\end{example}

A representation is said to be \emph{faithful} if it is injective. 

\begin{example}
  \index{Adjoint representation}
  Let $L$ be a Lie algebra. The \emph{adjoint representation} is the
  homomorphism $\ad\colon L\to\gl(L)$,
  $(\ad x)(y)=[x,y]$. 
\end{example}

If $L=\sl(2,\C)$, then $Z(L)=\{0\}$ (see Exercise~\ref{xca:center_sln}). In
particular, as $\ker\ad=Z(L)$, the adjoint representation is
faithful. In the basis $\{h,e,f\}$, 
\[
  \ad h=\begin{pmatrix}
    0&0&0\\
    0&2&0\\
    0&0&-2
  \end{pmatrix},
  \quad 
  \ad e=\begin{pmatrix}
    0&0&1\\
    -2&0&0\\
    0&0&0
  \end{pmatrix},
  \quad 
  \ad f=\begin{pmatrix}
    0&-1&0\\
    0&0&0\\
    2&0&0
  \end{pmatrix}.
\]

\begin{definition}
  \index{Module}
  Let $L$ be a Lie algebra. A (left) \emph{$L$-module} 
  is a finite-dimensional vector space $V$ together with a bilinear map 
  $L\times V\to V$, $(x,v)\mapsto x\cdot v$, such that 
  \[
    [x,y]\cdot v=x\cdot (y\cdot v)-y\cdot (x\cdot v)
  \]
  for
  all $x,y\in L$ and $v\in V$. 
\end{definition}

There is a bijective correspondence between representations of $L$ and
$L$-modules. If $\rho\colon L\to\gl(V)$ is a representation, then $V$ together
with $L\times V\to V$, $(x,v)\mapsto x\cdot v=\rho(x)(v)$, is an $L$-module.
Conversely, if $V$ is an $L$-module, then the map $\rho\colon L\to\gl(V)$,
$x\mapsto \rho_x$, where $\rho_x\colon V\to V$, $v\mapsto x\cdot v$, is a
representation. 

\begin{definition}
  Let $L$ be a Lie algebra and $V$ be an $L$-module. A subspace
  $W$ of $V$ is a \emph{submodule} of $x\cdot V\subseteq V$ 
  for all $x\in L$. 
\end{definition}

If $V$ is a module, $\{0\}$ and $V$ are submodules of $V$. 

\begin{definition}
  Let $L$ be a Lie algebra and $V$ and $W$ be $L$-modules.
  A linear map $f\colon V\to W$ is a module \emph{homomorphism} 
  if $f(x\cdot v)=x\cdot f(v)$ for all $x\in L$ and $v\in V$. 
\end{definition} 

Easy examples of module homomorphisms are the zero map and the identity. Of
course, compositions of module homomorphisms are module homomorphisms. 

\begin{example}
  Let $L$ be a Lie algebra. Then $L$ is an $L$-module with the adjoint
  representation. Submodules of $L$ are exactly the ideals of $L$. 
\end{example}

We write
\[
  \Hom_L(V,W)=\{f\colon V\to W:f\text{ is a module homomorphism}\}.
\]

If $f\colon V\to W$ is a module homomorphism, the \emph{kernel} $\ker f=\{v\in
V:f(v)=0\}$ of $f$, and the \emph{image} $f(V)$ of $f$ are submodules of $V$.
Moreover, the first isomorphism theorem holds, namely $V/\ker f\simeq f(V)$. 

For a Lie algebra $L$, $\Mod{L}$ denotes the category whose objects are
$L$-modules and morphisms are $L$-module homomorphisms. $\FgMod{L}$ is the full
subcategory of $\Mod{L}$ consisting of all finitely generated $L$-modules.

\begin{exercise}
  Let $L$ be a Lie algebra and $V$ and $W$ be $L$-modules. Prove 
  the following statements:
  \begin{enumerate}
    \item $V\oplus W=\{(v,w):v\in V,w\in W\}$ is an $L$-module
      with 
      \[
        x\cdot (v,w)=(g\cdot v,g\cdot w).
      \]
    \item $V\otimes W$ is an $L$-module with 
      \[
        x\cdot v\otimes w=(x\cdot v)\otimes w+v\otimes (x\cdot w).
      \]
  \end{enumerate}
\end{exercise}

If $V$ is an $L$-module, $\{0\}$ and $V$ are submodules of $V$. If
$W$ is a submodule of $V$, then $V/W$ is an $L$-module with $x\cdot
(v+W)=x\cdot v+W$.

\begin{definition}
  \index{Module!simple}
  Let $L$ be a Lie algebra and $V$ a non-zero 
  $L$-module. We say that $V$ is \emph{simple} if 
  $\{0\}$ and $V$ are the only submodules of $V$. 
\end{definition}

One-dimensional modules are simple. 

\begin{example}
  Let $L$ be a simple Lie algebra. Then $L$ viewed as an $L$-module with the
  adjoint representation is simple. 
\end{example}

\index{Representation!irreducible}
A representation is \emph{irreducible} if and only if the corresponding module
if simple. 

\begin{definition}
  \index{Module!indecomposable}
  Let $L$ be a Lie algebra and $V$ an $L$-module. We say 
  that $V$ is {indecomposable} if $V\simeq M\oplus N$ 
  as $L$-modules implies $M=\{0\}$ or $N=\{0\}$. 
\end{definition}

Simple modules are indecomposable. The converse is not true. 

\begin{definition}
  \index{Module!completely reducible}
  Let $L$ be a Lie algebra and $V$ an $L$-module. We say
  that $V$ is \emph{completely reducible} if 
  $V=S_1\oplus\cdots\oplus S_k$ for simple $L$-modules 
  $S_1,\dots,S_k$. 
\end{definition}

\begin{example}
  Let $L=\mathfrak{b}(2,\C)$ be the Lie algebra of $2\times 2$ upper triangular
  matrices. Then $V=\C^{2\times 1}$ is an $L$-module with the left action given
  by left multiplication.

  Let $e_1,e_2$ be the standard basis of $V$.  We claim that $\{0\}$, $\langle
  e_1\rangle$ and $V$ are the only submodules of $V$. From this, it follows
  that $V$ is indecomposable.  Let $W$ be a non-zero proper submodule of $V$.
  Then $\dim M=1$. Let $w=xe_1+ye_2$ be a basis of $W$. Then
  \[ \begin{pmatrix}0&1\\
    0&0\end{pmatrix}\begin{pmatrix}x\\y\end{pmatrix}
    =\begin{pmatrix}y\\0\end{pmatrix}\in W  
    \implies 
    \begin{pmatrix}y\\0\end{pmatrix}=\lambda\begin{pmatrix}x\\y\end{pmatrix}.
  \]
  This implies that 
  $\lambda=0$ or $y=0$. In both cases, 
  $W=\langle e_1\rangle$. 

  Note that $\langle e_1\rangle$ is a non-zero proper submodule of $V$, so $V$
  is not simple. 
\end{example}

\begin{exercise}
  Let $n\geq2$ and $L=\mathfrak{b}(n,\C)$ be the Lie algebra of $n\times n$ 
  upper triangular matrices. Prove the following statements: 
  \begin{enumerate}
    \item $V=\C^{n\times1}$ is an $L$-module.
    \item Let $e_1,\dots,e_n$ be the standard basis of $\C^{n\times1}$.
      For each $k\in\{1,\dots,n\}$, the subspace 
      $W_k=\langle e_1,\dots,e_k\rangle$ is a submodule of $V$. 
    \item Every submodule of $V$ is equal to some $W_k$. 
    \item Each $W_k$ is indecomposable. 
    \item $V$ is not completely reducible.
  \end{enumerate}
\end{exercise}
