% \thispagestyle{plain}
% \section*{Introduction}

% The notes correspond to the master  
% course \textbf{Associative Algebra} of the 
% Vrije Universiteit Brussel, 
% Faculty of Sciences, 
% Department of Mathematics and Data Sciences. The course
% is divided into twelve two-hour lectures. 

% % \bigskip 
% % \textbf{Exercises}
% \subsection*{Mandatory and bonus exercises}

% The notes include many exercises, some with full detailed solutions. Mandatory exercises have a \colorbox{green!5!white}{green background}, while optional ones (bonus exercises) have a \colorbox{yellow!15!white}{yellow background}.

% \subsection*{Prerequisites and bibliography}

% The reader should have a solid understanding of an undergraduate-level abstract algebra course. For reference, you can review my notes for the VUB courses: \href{https://github.com/vendramin/group}{Group Theory} and \href{https://github.com/vendramin/rings}{Ring and Module Theory}.

% The content presented here draws heavily from \cite{MR3308118}, \cite{MR1449137}, 
% and \cite{MR798076}. Additionally, I have 
% followed the outstanding \href{https://ysharifi.wordpress.com}{blog} on 
% abstract algebra by Yaghoub Sharif.


% \subsection*{Final projects} 

% The document contains various topics for final projects, including some that could be expanded into bachelor's or master's theses. Here are some additional topics.

% \subsection*{Rickart's theorem}

% In Lecture \ref{09} we presented an algebraic proof of Rickart's theorem. 
% The original proof uses analysis; see Appendix \ref{section:Rickart} or \cite[(6.4) of Chapter II]{MR1838439}. 

%\subsubsection*{Connel's theorem}
%
%In Lecture \ref{11} we presented the statement of Connel's theorem, which
%characterizes prime group rings over fields of characteristic zero 
%(see Theorem \ref{thm:Connel}); the proof of this  
%result appears for example in \cite[Theorem 2.10 of Chapter 4]{MR798076}. 
%As a corollary, one obtains 
%that, if $K$ is a field of characteristic zero,
%then the group ring $K[G]$ is left artinian if and only if the group
%$G$ is finite; see 
%\cite[Theorem 1.1 of Chapter 10]{MR798076} for a proof. 

% \subsubsection*{Kolchin's theorem}

% Let $U_n(\C)$ be the subgroup of $\GL_n(\C)$ 
% of matrices $(u_{ij})$ such that 
% \[
% u_{ij}=\begin{cases}
% 1&\text{if $i=j$},\\
% 0&\text{if $i>j$}.\end{cases}
% \]

% A matrix $a\in\GL_n(\C)$ is said to be \emph{unipotent} 
% if its characteristic polynomial is of the form $(X-1)^n$. 
% A subgroup $G$ of $\GL_n(\C)$ is said to be \emph{unipotent} if
% each $g\in G$ is unipotent. 

% An important theorem of Kolchin states that 
% every unipotent subgroup of $\GL_n(\C)$ is conjugate
% of some subgroup of $U_n(\C)$. The theorem and its proof 
% appear, for example, 
% in the 
% VUB course \href{https://github.com/vendramin/representation}{Representation theory of algebras}.


\thispagestyle{plain}
\section*{Introduction}

The notes correspond to the master  
course \emph{Lie Algebras} of the 
Vrije Universiteit Brussel, 
Faculty of Sciences, 
Department of Mathematics and Data Sciences. The course
is divided into twelve two-hour lectures. 

The reader should have a solid understanding of an undergraduate-level abstract
algebra course. For reference, you can review my notes for the VUB courses:
\href{https://github.com/vendramin/group}{Group Theory} and
\href{https://github.com/vendramin/rings}{Ring and Module Theory}.

The content presented here draws heavily from X, Y 
and \cite{MR1321145}. 

The notes include many exercises, some with full detailed solutions. Mandatory exercises have a \colorbox{green!5!white}{green background}, while optional ones
(bonus exercises) have a \colorbox{yellow!15!white}{yellow background}.

The notes also include some additional comments. While these are entirely optional, I hope they offer further insight. They are highlighted with a \colorbox{red!5!white}{pink background}.

The notes include Magma code, which we use to verify examples and offer alternative solutions to certain exercises. Magma \cite{zbMATH01077111} is a powerful software tool designed for working with algebraic structures. There is a free \href{https://magma.maths.usyd.edu.au/calc/}{online} version of Magma available.


This version 
was compiled on \today~at~\currenttime.

 \begin{figure}[b]
     \includegraphics[scale=0.2]{VUB.jpg}
 \end{figure}


